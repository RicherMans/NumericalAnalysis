\section{Exercise 1}
We want to prove that $\rho ( I - Q^{-1} A ) < 1$. To do this, we will show that $\max \det( I - Q^{-1} A ) < 1$, which means that the eigenvalues $\max_{i} (\lambda_i) < 1 $. 
First we use the assumptions of having a Jacobian method to set $Q=D$, where $D = \text{diag} (A)$ is diagonal. Moreover since $Q$ is diagonal, $Q^{-1}$ of element $q^{-1}_{ij} = \frac{1}{q_{ij}}$. Moreover we can see from the definition that $q_{ii} = \frac{1}{a_{ii}}$.

From that on follows:
\begin{gather*}
\det(\lambda I - I - Q^{-1} A) = 0  \\
= \det \left(
\left( \begin{array}{ccc}
\lambda_1 & 0 & 0 \\
0 & \ddots & 0\\
0 & 0 & \lambda_n
\end{array} \right)
-
\left( \begin{array}{ccc}
1 & 0 & 0 \\
0 & \ddots & 0\\
0 & 0 & 1
\end{array} \right)
-
\left( \begin{array}{ccc}
\dfrac{1}{a_{11}} & 0 & 0 \\
0 & \ddots & 0\\
0 & 0 & \dfrac{1}{a_{nn}}
\end{array} \right)
\left( \begin{array}{ccc}
a_{11} & \hdots & a_{1n} \\
\vdots & \ddots & \vdots\\
a_{n1} & \hdots & a_{nn}
\end{array} \right)
\right) 
= \\
\left( \begin{array}{ccc}
\lambda_1 - 1 & \dfrac{1}{a_{11}}a_{1k} & \dfrac{1}{a_{11}} a_{1n} \\
\dfrac{1}{a_{kk}}a_{k1} & \ddots & \dfrac{1}{a_{kk}}a_{1n} \\
\dfrac{1}{a_{nn}}a_{n1} & \dfrac{1}{a_{nn}}a_{kk} & \lambda_n - 1
\end{array} \right) = B
\end{gather*}
When calculating the determinant:
\begin{gather*}
\det(A) = \sum_{i=1}^n (-1)^{i+j} a_{i,j} M_{i,j} = 0 \\
\begin{cases}
      a_{i,j} = 0 , & \text{if}\ \lambda_i \leq 1 \\
      M_{i,j} = 0 & \text{if}\ \lambda_i \leq 1
 \end{cases}
\end{gather*}
Since the resulting matrix is still diagonally dominant, because we only subtracted and added values onto the diagonal, it must be that our resulting matrix $B$, still is strictly diagonally dominant.
That means that $|b_{ii}| \geq \sum_{j\neq i} |b_{ij}| \quad\text{for all } i$, so that follows : $\max_i (\lambda_i) < 1$, which means that $\rho ( I - Q^{-1} A ) < 1$.

\section{Exercise 2}
The Richardson Iteration is stated as : $ x^k = (I-A) x^{k-1} +b$. The Richardson iteration will be successful if $||I-A|| < 1 $.We have:
\begin{gather*}
I - A = 
\left( \begin{array}{ccc}
1 & 0 & 0\\
0 & \ddots & 0\\
0 & 0 & 1\\
\end{array}\right) 
-
\left( \begin{array}{ccc}
1 & a_{1k} & a_{1n}\\
a_{k1} & \ddots & \vdots \\
a_{n1} & \hdots & 1\\
\end{array}\right)
= 
\left( \begin{array}{ccc}
0 & -a_{1k} & -a_{1n}\\
-a_{k1} & \ddots & \vdots \\
-a_{n1} & \hdots & 0\\
\end{array}\right)
\end{gather*}
Since A is diagonally dominant, $\sum\limits_i |a_{ij}| < 1 $, after the subtraction of both matrices we get $0 - \sum\limits_i |a_{ij}| < 1 $, which only reverses the sign, but this fact doesn’t matter, since we use the absolute value, so that still $||I-A|| < 1$.
\section{Exercise 5}
We need to prove that $ ||x||^{'} = ||Sx||$ is a norm.
\begin{enumerate}
\item Prove that $||x||^{'} > 0$. Since it is assumed that $S$ is non-singular, we know that if we would modify $S$ to be in Echelon form, we would not get any row $r$, which has at it's $r$th column entry a zero value, therefore when we multiply $S$ with $x$, the resulting value would be $>0$.
\item To show that $||\lambda S x || \rightarrow |\lambda| ||S x||$. Assuming having $\lambda >0$ we can easily see that this equation holds, since the matrix multiplication with a scalar is commutative.
\item To show that the triangle inequality holds, we see that $|| x + y || = \sup{||S_1 x + S_2 y||} \leq \sup{|| S_1 x ||} + \sup{|| S_2 y||} $. It follows : $|| x+y || \leq ||x|| + ||y||$.
\end{enumerate}

\section{Exercise 7}

We will prove that $||I-Q^{-1}A||_\infty < 1 $, which differently interpreted means $  \left \|  I-Q^{-1}A \right \| _\infty = \max \limits _{1 \leq i \leq m} \sum _{j=1} ^n | a_{ij} |$.
$Q $ is a Gauss Seidel iteration variable. $Q = D-L$, $A = D - L - U$ and $ D - L = A + U$
We expand:
\begin{gather*}
I - (A + U ) ^{-1} A  = \dfrac{(A+U)(I - (A + U ) ^{-1} A )}{A+U} = \dfrac{(A+U) - A}{A+U} = \dfrac{U}{A+U}
\end{gather*}
Now it is sufficient to show that $||U|| < ||A+U||$.
\begin{gather*}
A+U = \left( \begin{array}{cccc}
a_{11} & 2 a_{1k} & 2 a_{1k+1} & 2 a_{1n} \\
\vdots & \ddots & & 2 a_{kn}  \\
\vdots & & \ddots & 2 a_{k+1n}\\
a_{n1} & \hdots& \hdots & a_{nn} 
\end{array} \right)
\end{gather*}
So it can be seen that the upper triangular elements in $A+U$ do double, whereas all other elements remain the same. This means that $\max_i \sum\limits_i |u_{ij}| < \max_i \sum\limits_i |ua_{ij}| $. Since $||U|| > 0 $ and therefore $||A+U|| > 0 \rightarrow ||I - Q^{-1}A|| < 1 $.
\section{Exercise 8}
We want to show that $\left(\lim_{k \to \infty}A^k = 0 \Rightarrow \rho(A) < 1\right)$.
Moreover since $\rho$ is defined as : $\rho(A) = \max_i(|\lambda_i|)$, we need only to show that all eigenvalues of $\left(\lim_{k \to \infty} A^k \right) < 1 $. For any eigenvector $v$, we get via eigenvector definition:
\begin{gather*}
A^k v = \lambda^k v \\
0 = \left(\lim_{k \to \infty}A^k \right) v = 
\lim_{k \to \infty}A^k v = 
\lim_{k \to \infty} \lambda^k v  = v \lim_{k \to \infty} \lambda^k
\end{gather*}
Per definition $v \neq 0$, it is obvious that $\lim_{k \to \infty} \lambda^k = 0$. This fact already implies $|\lambda| < 1$.

\section{Exercise 15}
Let $\lambda$ be an eigenvalue of $ I - Q^{-1}A$ and $x$ the corresponding eigenvectors with $||x||_{\infty} = 1$.
We get:
\begin{gather*}
\left( Q -A \right) x = \lambda Q x  \\
\lambda a_{ii}x_i = - \lambda \sum\limits_{j=1}^{i-1} a_{ij}x_{j} - \sum\limits_{j=i+1}^{n} a_{ij}x_j \\
|x_i| = 1 \geq ||x_j|| \forall j\\
|\lambda | |a_{ii}| \leq  |\lambda| \sum\limits_{j=1}^{i-1} |a_{ij}| + \sum\limits_{j=i+1}^{n} |a_{ij}| \\
\text{knowing that } |a_{ii}| > \sum\limits_{j,j \neq i}^n |a_{ij}| = \sum\limits_{j=i+1}^{n} |a_{ij}| + \sum\limits_{j=1}^{i-1} |a_{ij}| \\
|\lambda| |a_{ii}| - |\lambda| \sum\limits_{j=i+1}^{n} |a_{ij}| \leq \sum\limits_{j=1}^{i-1} |a_{ij}|  \\
|\lambda| ( |a_{ii}| - \sum\limits_{j=1}^{i-1} |a_{ij}| ) \leq \sum\limits_{j=i+1}^{n} |a_{ij}|
\\ 
|\lambda| \leq r_i < 1 , \text{ where }\ 
r_i = \dfrac{ \sum\limits_{j=i+1}^{n} |a_{ij}| }{ |a_{ii}| - \sum\limits_{j=1}^{i-1} |a_{ij}| }
\end{gather*}
This shows that $\rho(I-Q^{-1}A)$ is not greater than $r_i$.

\section{Exercise 20}
We need to prove that if $\rho(A) < 1$, $(I-A)^{-1}$ exists and $\sum\limits_k^\infty A^k = (I-A)^{-1}$.
First we know that the formula to calculate the eigenvalues is $\det ( \lambda I - A) =0$. In other words, we seek for these $\lambda_i$s ,which will lead the term $\lambda I -A$ to become not invertible. Since we know that $\rho (A) < 1$, which means that all eigenvalues $\lambda_i$ are less than 1. We come to the conclusion, that $(\beta I - A)$ is invertible, if $\beta_i \neq \lambda_i$. In other words, if we choose $\beta$ as $I$, we can guarantee that $(I-A)$ is invertible, since all the necessary eigenvalues in the diagonal of $I$ need to be less than 1.

Now we know that $(I-A)$ is invertible, so we can use this knowledge to modify the equations and get:
\begin{equation}
\label{eq:1}
(I-A)^{-1} = \sum\limits_k^{\infty} A^k \\
I =\sum\limits_k^{\infty} A^k (I-A) = (I-A) \sum\limits_k^{\infty} A^k \\
\sum\limits_k^{\infty} \left( A^k- A^{k+1} \right)
\end{equation}
Since all the terms in the equation \ref{eq:1} cancel itself out, except the first and the last one, we get:
\begin{gather*}
I - A^{m+1} \text{ where }\ m \rightarrow \infty
\end{gather*}
Again by using our basic assumption that $\rho(A) < 1$, we show that the matrix norm is less than 1. For an eigenvector $v$, we have the following equation based on the eigenvalue extraction.
\begin{align}
\label{eq:2}
|\lambda|^k\|v\| = \|\lambda^k \mathbf{v}\| = \|A^k v\| \leq \|A^k\|\cdot\|v\|
\rightarrow
|\lambda|^k\leq \|A^k\|
\end{align}
Since our maximum eigenvalue is less than 1, $|\lambda|^k$ is surely less than 1 and therefore $||A^k||$ is too.
This fact leads to that $A^{m+1}$ in equation \ref{eq:2} will converge to zero, so that $I$ will be the result of the series.
This proofs that $(I-A)$ is invertible and has a series expansion.

\section{Exercise 30}
All matrices follow a general approach, that they seek a splitting matrix $Q$, so that $Qx = (Q-A)x + b$ can iteratively be solved.
This follows into the basic iteration equations.
\begin{equation}
\label{eq:3}
Qx^k = (Q-A)x^{k-1} + b
\end{equation}
\begin{equation}
\label{eq:4}
x^k = G x^{k-1} + c
\end{equation}

\paragraph{Richardson}
For the Richardson iteration, we choose $Q$ as $I$, so that we get:
\begin{gather*}
Q = I \\
x^k = (I-A)x^{k-1} + b \\
\mathcal{R} = (I - A) = G \\
\end{gather*}
The matrix R is the iteration matrix and uniquely also $G$.

\paragraph{Jacobi}
For the Jacobi iteration, we choose $Q$ as $D$, so that we get for $G$:
\begin{gather*}
Dx^k = (D-A)x^{k-1} + b \\
Dx^k = (D-(D-L-U)) x^{k-1} + b \\
Dx^k = (L+U) x^{k-1} + b\\
x^k = D^{-1}(L+U) x^{k-1} + D^{-1} b \\
\rightarrow G = D^{-1}(L+U)
\end{gather*}
And to show the iteration matrix:
\begin{gather*}
Dx^k = (D-A)x^{k-1} + b \\
x^k = D^{-1}(D-A) x^{k-1} + D^{-1}b \\
x^k = I - D^{-1}A x^{k-1} + D^{-1}b\\
\rightarrow \mathcal{J} = I - D^{-1}A
\end{gather*}

\paragraph{Gauss Seidel}
For the Gauss Seidel iteration, we choose $Q$ as $D-L$, so that we get:
\begin{gather*}
(D-L)x^k = (D-L-A) x^{k-1} + b \\
(D-L)x^k = (D-L-(D-L-U)) x^{k-1} + b \\
x^k = (D-L)^{-1}U x^{k-1} + (D-L)^{-1}b \\
\rightarrow G = (D-L)^{-1}U
\end{gather*}
To show that the iteration matrix is correct:
\begin{gather*}
(D-L)x^k = (D-L-A) x^{k-1} + b \\
x^k = (D-L)^{-1} (D-L-A) x^{k-1} + (D-L)^{-1}b\\
x^k = (D-L)^{-1} (D-L) - (D-L)^{-1} A x^{k-1} + (D-L)^{-1}b\\
x^k = I - (D-L)^{-1} A x^{k-1} + (D-L)^{-1}b \\
\rightarrow \mathcal{G} = I - (D-C_L)^{-1} A , \text{where}\ L = C_L
\end{gather*}

\paragraph{Forward SOR}
For the forward SOR method, we choose $Q$ as $\omega^{-1}(D-\omega C_L) $, so we get:
\begin{gather*}
\omega^{-1}(D-\omega C_L) x^{k} = (\omega^{-1}(D-\omega C_L) -A) x^{k-1} + b \\
\omega^{-1} x^{k} = (D-\omega C_L)^{-1}(\omega^{-1}(D-\omega C_L) -A) x^{k-1} + (D-\omega C_L)^{-1} b \\
\omega^{-1} x^{k} = \omega^{-1}(D-\omega C_L)^{-1}(D-\omega C_L) - (D-\omega C_L)^{-1}A x^{k-1} + ((D-\omega C_L))^{-1} b \\
\omega^{-1} x^{k} = \omega^{-1}I - (D-\omega C_L)^{-1}A x^{k-1} + ((D-\omega C_L))^{-1} b \\
x^{k} = \omega \omega^{-1}I - \omega (D-\omega C_L)^{-1}A x^{k-1} + \omega((D-\omega C_L))^{-1} b \\
\rightarrow \mathcal{L_{\omega}} = I - \omega (D-\omega C_L)^{-1}A
\end{gather*}
To find out $G$, we get:
\begin{gather*}
D(x^{k} - x^{k-1}) =  \omega b - \omega D x^{k-1} + \omega L x^{k} + \omega U x^{k-1}\\
(D - \omega L) x^{k} = Dx^{k-1} -  \omega D x^{k-1} + \omega U x^{k-1} + \omega b\\
(D - \omega L) x^{k} = (D - \omega D + \omega U ) x^{k-1} + \omega b \\
x^{k} =  (D - \omega L)^{-1} (D - \omega D + \omega U ) x^{k-1} + (D - \omega L)^{-1} \omega b \\
\rightarrow G = (D - \omega L)^{-1} (D - \omega D + \omega U )
\end{gather*}
\paragraph{Backward SOR}
Since Backward and Forward SOR are only a manipulation of $C_L = C_U$, replace the variables and it can be easily seen that the result is the same, except for this variable.

\begin{gather*}
G = (D - \omega C_R)^{-1} (D - \omega D + \omega U ) \\
\mathcal{U_{\omega}} = I - \omega (D-\omega C_R)^{-1}A
\end{gather*}

\paragraph{SSOR}

In SSOR, our splitting matrix $Q = (\omega (2 - \omega)^{-1} (D-\omega C_L)D^{-1}(D-\omega C_U)$
\begin{gather*}
Qx^k = (Q-A) x^{k-1} + b \\
x^{k} = Q^{-1}(Q-A) x^{k-1} + Q^{-1}b\\
x^k = (\omega (2 - \omega) (D-\omega C_L)^{-1} D(D-\omega C_U)^{-1} (\omega (2 - \omega)^{-1} (D-\omega C_L)D^{-1}(D-\omega C_U)-\\ (\omega (2 - \omega) (D-\omega C_L)^{-1} D(D-\omega C_U)^{-1} A x^{k-1} + Q^{-1}b \\
x^k = I - (\omega (2 - \omega) (D-\omega C_L)^{-1} D(D-\omega C_U)^{-1} A x^{k-1} + Q^{-1} b \\
\rightarrow \mathcal{S_{\omega}} =  I - (\omega (2 - \omega) (D-\omega C_L)^{-1} D(D-\omega C_U)^{-1} A
\end{gather*}
By expanding the terms $A$ and $C_U$, like in the SOR iteration, we get:
%Totaler bullshit hier
\begin{gather*}
x^k =  I - (\omega (2 - \omega) (D-\omega C_L)^{-1} D(D-\omega C_U)^{-1} A x^{k-1}\\
x^k = (D-\omega C_U)^{-1} (\omega C_L + ( 1- \omega)D ) (D-\omega C_L)^{-1}(\omega C_U + (1-\omega D) x^{k-1} \\
\rightarrow G = (D-\omega C_U)^{-1} (\omega C_L + ( 1- \omega)D ) (D-\omega C_L)^{-1}(\omega C_U + (1-\omega D) x^{k-1} 
\end{gather*}


\section{Exercise 31}
In this exercise we need to find the explicit form of $I-Q^{-1}A$.
\begin{gather*}
A = \left(\begin{array}{cccccc}
2 & -1 &  & & & \\
-1 & 2 & -1 &  & &  \\
 & -1 & 2 & -1 & &  \\
 & & \ddots & \ddots & \ddots & \\
  & & &   -1 & 2 & -1\\
 & & & &  -1 & 2
\end{array} \right)
Q = \left( \begin{array}{cccccc}
2 & 0 &  & & & \\
1 & 2 & 0 &  & &  \\
 & 1 & 2 & 0 & &  \\
 & & \ddots & \ddots & \ddots & \\
  & & &   1 & 2 & 0\\
 & & & &  1 & 2
\end{array} \right) \\
\end{gather*}
Since $Q$ is lower triagonal, the determinant is easy to find. To get the adjugate matrix, we can show that it will be $2^{n-1}$ for every diagonal entry and will decrease in for each step farer away from the diagonal.
\begin{gather*}
Q^{-1} = \left( \begin{array}{cccccc}
2^n & 0 &  & & & \\
-2^{n-1} & 2^n & 0 &  & &  \\
2^{n-2} & -2^{n-1} & 2^n & 0 & &  \\
 & \ddots & \ddots & \ddots & \ddots & \\
2^{n-4}  & -2^{n-3} & 2^{n-2} &   -2^{n-1} & 2^n & 0\\
 & 2^{n-4} & -2^{n-3} & 2^{n-2} &  -2^{n-1} & 2^n
\end{array} \right)
\\
Q^{-1} A = 
\left( \begin{array}{cccccc}
2^{n+1} + 2^{n-1} & -2^{n} &  & & & \\
-2^n - 2^n & 2^{n+1} + 2^{n-1} & -2^{n} &  & &  \\
2^{n-1} + 2^{n-1} & -2^n - 2^n & 2^{n+1} +2^{n-1} & -2^{n} & &  \\
-2^{n-2} - 2^{n-2} & \ddots & \ddots & \ddots &  \\
  &   &   -2^n - 2^n & 2^{n+1} +2^{n-1} & -2^{n}\\
 &   -2^{n-2} - 2^{n-2} & 2^{n-1} + 2^{n-1} &  -2^n - 2^n & 2^{n+1} + 2^{n-1}
\end{array} \right)
\\
I - Q^{-1}A = 
\left( \begin{array}{cccccc}
1 - 2^{n+1} + 2^{n-1} & 2^{n} &  & & & \\
2^n + 2^n & 1 - 2^{n+1} + 2^{n-1} & 2^{n} &  & &  \\
-2^{n-1} - 2^{n-1} & 2^n + 2^n & 1 - 2^{n+1} +2^{n-1} & 2^{n} & &  \\
2^{n-2} + 2^{n-2} & \ddots & \ddots & \ddots &  \\
  &   &   2^n + 2^n & 1 - 2^{n+1} +2^{n-1} &  2^{n}\\
 &   2^{n-2} + 2^{n-2} & - 2^{n-1} - 2^{n-1} &  2^n + 2^n & 1- 2^{n+1} + 2^{n-1}
\end{array} \right)
\end{gather*}
Which is a closed form solution.

\section{Exercise 35}
We want to show that $||x^k -x || = \dfrac{\delta}{1-\delta} || x^k - x^{k-1} ||$.
We set $d(x_m,x_n) = ||x^k -x ||$, so that $x_n$ is a fixed point and $x_m$ is an iteration.
Assuming $m > n$:
\begin{gather*}
d(x_m, x_n)  \leq d(x_m, x_{m-1}) + d(x_{m-1}, x_{m-2}) + \cdots + d(x_{n+1}, x_n)   \\
 \leq q^{m-1}d(x_1, x_0) + q^{m-2}d(x_1, x_0) + \cdots + q^nd(x_1, x_0) \\
 = q^n d(x_1, x_0) \sum_{k=0}^{m-n-1} q^k \\
 \leq q^n d(x_1, x_0) \sum_{k=0}^\infty q^k \ \text{ since } q < 1\\
 = q^n d(x_1, x_0) \left ( \frac{1}{1-q} \right ) 
\end{gather*}
If we choose $n$ as being equal to $m$, this equation simplifies to $||x^k -x || = \dfrac{\delta}{1-\delta} || x^k - x^{k-1} ||$, as required.
