\section{Exercise 6}
We need to determinate whether $f(x)$ is a quadratic spline.
Therefore we must verify whether $f \in C^2(R)$ or $f \in C^3(R)$.

\begin{gather*}
f' (x) = \begin{cases}
1 \quad\quad\quad x \in (-\infty,1] \\
2-x \quad\quad\quad x \in [1,2] \\
0 \quad\quad\quad x\in [2,\infty)
\end{cases},
f''(x) =\begin{cases}
0 \quad\quad\quad x \in (-\infty,1] \\
-1 \quad\quad\quad x \in [1,2] \\
0 \quad\quad\quad x\in [2,\infty) \\
\end{cases}
\end{gather*}
Now we use the limits between two of the spline cases to calculate whether the functions are quadratic and so on.
\begin{gather*}
\lim_{x \to 1^{-}} f(x) = 1 =  \lim_{x \to 1^{+}} f(x) , \lim_{x \to 2^{-}} f(x) = \frac{3}{2} = \lim_{x \to 2^{+}} f(x)
\end{gather*}
hence $f \in C^1(R)$
\begin{gather*}
\lim_{x \to 1^{-}} f'(x) = 1 =  \lim_{x \to 1^{+}} f'(x) , \lim_{x \to 2^{-}} f'(x) = 0 = \lim_{x \to 2^{+}} f'(x)
\end{gather*}
hence $f \in C^2(R)$.
\begin{gather*}
\lim_{x \to 1^{-}} f''(x) = 0, \lim_{x \to 1^{+}} f''(x) = -1 , \lim_{x \to 2^{-}} f''(x) = -1 , \lim_{x \to 2^{+}} f''(x) =0
\end{gather*}
Hence $f \not \in C^3(R)$, so we can see that $f$ is a quadratic, but not a cubic spline.
\section{Exercise 7}
As already seen in paragraph $6$, it is not a cubic spline.
\section{Exercise 8}

The given functions and its derivatives are:
\begin{gather*}
f(x) = \begin{cases}
a(x-2)^2 + b(x-1) ^3 \quad\quad x \in (-\infty,1]\\
c(x-2)^2 \quad\quad x\in [1,3]\\
d(x-2)^2 + e(x-3)^3 \quad \quad x \in [3,\infty)
\end{cases}\\
f(x)' =\begin{cases}
2a(x-2) + 3b(x-1)^2 \quad \quad x \in (-\infty,1]\\
2c \quad\quad x\in [1,3]\\
2d(x-2) + 3e(x-3)^2 \quad \quad x \in [3,\infty)
\end{cases}\\
f(x)''=\begin{cases}
2a + 6b(x-1) \quad \quad x \in (-\infty,1]\\
2c \quad\quad x\in [1,3]\\
2d + 6e(x-3) \quad \quad x \in [3,\infty)
\end{cases}
\end{gather*}
To determinate the spline we first use the knot conditions:
\begin{gather*}
S_{i-1}(t_i) = S_{i}(t_i) \\
S_0(1) =  = S_1(1) =\\
a(x-2)^2 = c(x-2)^2\\
= a = c
\end{gather*}
Moreover we get:
\begin{gather*}
S_1(3) = S_2(3) \\
= -c = -d \\
c = d \Rightarrow a = c = d
\end{gather*}
Interestingly, we didn't use the derivatives as conditions, since they actually give us exactly the same results for the knots.
$f(x)$ is a cubic spline, if and only if $a = c = d$, the parameters $b,e$ are arbitrary chosen.
Now we calculate the parameters, using $f(0) = 26$ in equation $S_0$:
\begin{gather*}
4a - b = 26\\
\end{gather*}
Using $f(1) = 7$ in $S_0$:
\begin{gather*}
a+0  = 7 \Rightarrow a = 7 = c = d \Rightarrow b = 4a -26 = 2
\end{gather*}
Lastly using $f(4) = 25$, in $S_2$.
\begin{gather*}
4d + e = 25\\
28 + e = 25\\
e= -3 
\end{gather*}
So our result is : $a = c =d = 7$, $b =2$, $e = -3$.


