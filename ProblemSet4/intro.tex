\section{Exercise 2}
We prove that $x_n = ( 1 + \dfrac{1}{n} ) ^n$ converges to $e$ in its limits.
The proof assumes that some real function exists, $e^x$, which is monotonic, so that it's inverse $\log(x)$ also has a monotonic growth.

\begin{gather*}
\lim_{n \to \infty} \left( 1 + \frac{1}{n} \right)^n = \lim_{n \to \infty} \left( e^{\log(1 + \frac{1}{n} )^n} \right) = e^{ \lim_{n \to \infty} \left( \log(1+\frac{1}{n} \right)^n } \\
=  e^{ \lim_{n \to \infty} \left( n \log(1+\frac{1}{n} \right)} = e^{ \lim_{n \to \infty} \left( \dfrac{\log \left( 1 + \frac{1}{n} \right)}{\frac{1}{n}} \right) }  \\
= e^{ \lim_{n \to \infty} \left( \dfrac{\frac{1}{1+\frac{1}{n}} \left( - \frac{1}{n^2} \right)}{ -\frac{1}{n^2}} \right) } = e^{ \lim_{n \to \infty} \left( \frac{1}{1+\frac{1}{n}} \right) } = e^1 = e < 3
\end{gather*}
We have shown that the series $(1+\frac{1}{n})$ converges to a value lesser than $3$.

\section{Exercise 6}
We need to show that $ x_n = O(\alpha_n)  \rightarrow c x_n = O (\alpha_n)$.

We use the definition of big $O$, so we get:
\begin{gather*}
c x_n = O (\alpha_n) = \lim\limits_{n \to \infty} c \left| \dfrac{x_n}{\alpha_n}\right| < \infty\\ 
= c \lim\limits_{n \to \infty} \left| \dfrac{x_n}{\alpha_n} \right| < \infty \\
= \lim\limits_{n \to \infty} \left| \dfrac{x_n}{\alpha_n} \right| < \infty \\
= x_n = O(\alpha_n)
\end{gather*}
It is shown that this equation holds, since constants in cases of infinite calculations cancel out.
\section{Exercise 7}
We want to show that from $x_n = O(\alpha_n)$ follows $\dfrac{x_n}{\log(n)} = o(\alpha_n)$.
\begin{gather*}
\frac{x_n}{ln(n)} = o(\alpha_n) = \lim_{n \to \infty} \left| \frac{x_n}{\log(n) \alpha_n} \right| \\
=  \lim_{n \to \infty} \left| \frac{1}{\log(n) } \right| \left| \frac{x_n}{\alpha_n} \right| \\
\lim_{n \to \infty} \left| \frac{1}{\log(n) } \right| \lim_{n \to \infty } \left|\frac{x_n}{\alpha_n} \right| \leq C \rightarrow O(\alpha_n)
\end{gather*}
Since the equation $  \frac{x_n}{\alpha_n} \leq C $ is bounded by $C$, yet converges to 0, if $n \to \infty$, we can see that the factor $\log (n)$ only increases its convergence.
\section{Exercise 8}
We need to find the best value for $k$ in the term $\cos(x) -1 + \frac{x^2}{2} = O(x^k) $. We use the Taylor series $ \sum_{n=0} ^ {\infty} \dfrac {f^{(n)}(a)}{n!} \, (x-a)^{n}$ here to estimate $\cos(x)$. We use for convenience $a=0$.
\begin{gather*}
\cos (x) -1 + \frac{x^2}{2} = O(x^k)\\
1 - \frac{1}{2!} x^2 + \frac{1}{4!} x^4 - \frac{1}{6!} x^6 + \frac{1}{8!} x^8 \ldots - 1 +\frac{x^2}{2} \\
= \frac{1}{4!} x^4 - \frac{1}{6!} x^6 + \frac{1}{8!} x^8 \ldots = O(x^k) \rightarrow k = 4
\end{gather*}
For the value of $k = $, we can see that, as long as $x \to 0$, the terms after the $x^4$ are always smaller than $C x^4$, so that $x^4$ is an upper bound.
\section{Exercise 12}
We show that for any $r > 0, x^r = O(x^r)$.
\begin{gather*}
\lim_{x \to \infty} \dfrac{x^r}{e^x} = \lim_{x \to \infty} \dfrac{r \log(x)}{x} = r \lim_{x \to \infty} \dfrac{\log(x)}{x}\\
\left| \lim_{x \to \infty} \dfrac{\log(x)}{x} \right| \leq \frac{C}{r} 
\end{gather*}
Which shows that this equation holds for every $r>0$.
\section{Exercise 13}
We show that for any $r > 0, \log(x) = O(x^r)$.
\begin{gather*}
\lim_{x \to \infty} \left| \dfrac{\log(x)}{x^r} \right| \\
\text{ we use Taylor series at a = 1 } \\
\log(x) = - \frac{1}{x}(x-1) + \frac{1}{x^2}(x-1)^2 + \underbrace{\ldots}_{\to 0}\\
\text{ If the denominator is always larger than the nominator, the fraction will converge to zero }\\
\log(x) = -1 + \frac{1}{x} + 1 - \frac{2}{x} + \frac{1}{x^2} \\
\lim_{x \to \infty} \underbrace{x^r}_{\to \infty} > \log(x) = \underbrace{-\frac{1}{x} + \frac{1}{x^2}}_{\to 0}
\end{gather*}
So the fraction $ \dfrac{\log(x)}{x^r} $ is bounded and will converge to zero.
\section{Exercise 16}
We need to determinate the best value for $k$ in the equation $\tan^{-1}(x) = x+ O(x^k)$, alternatively : $ O(x^k) = \tan^{-1}(x) -x$
\begin{gather*}
\tan^{-1}(x) = 1 - \frac{1}{3} x^3 + \frac{1}{5} x^5 - \frac{1}{7} x^7 \ldots \\
\tan^{-1}(x) -x = 1 - \frac{1}{3} x^3 + \frac{1}{5} x^5 - \frac{1}{7} x^7 \ldots -x \\
1 + x \left( -\frac{1}{3} x^2 + \frac{1}{5} x^5 - \frac{1}{7} x^7 \ldots - 1  \right) \\
\end{gather*}
As we can see the whole expression converges to one, if $x \to 0$. So we choose our $C$ so that $ 1 \leq C x^k $. If we choose $k = 0$, $C$ can be chosen $\geq 1$, so that the equation holds.
\section{Exercise 22}
In each $n \to \infty$.
\paragraph{a}
This assertion does not hold:
\begin{gather*}
\dfrac{n+1}{n^2} = o(\frac{1}{n}) \\
\lim_{n \to \infty} \dfrac{\dfrac{n+1}{n^2}}{\dfrac{1}{n}} \hat{=} 0 \\
= \lim_{n \to \infty} \dfrac{n^2 + n}{n^2} = \lim_{n \to \infty} 1+\frac{1}{n} = 1 \neq 0
\end{gather*}
The value of this equation is not bounded by 0, so it could be big O, but not a small o.
\paragraph{b}
This assertion does not hold either.
\begin{gather*}
\dfrac{n+1}{\sqrt[2]{n}} = o(1) \\
\lim_{n \to \infty} \dfrac{n+1}{\sqrt[2]{n}} = \lim_{n \to \infty} \sqrt[2]{n} + 1 = \infty \neq 0
\end{gather*}
\paragraph{c}
\begin{gather*}
\frac{1}{ln(n)} = O (\frac{1}{n}) \\
\lim_{n \to \infty} \dfrac{e^n}{n}  = \infty 
\end{gather*}
This term should approach any constant $L$, which is smaller than $\infty$, which it doesn't.
\paragraph{d}
This term holds the assertion
\begin{gather*}
\frac{1}{n \log n} = o(\frac{1}{n}) \\
\lim_{n \to \infty} \dfrac{n}{n \log(n)} =  \lim_{n \to \infty}  \dfrac{1}{\log n} = \lim_{n \to \infty} \dfrac{e}{n} = 0 
\end{gather*}
This term holds.
\paragraph{e}
This assertion does not hold
\begin{gather*}
\frac{e^n}{n^5} = O(\frac{1}{n})\\
\lim_{n \to \infty} \frac{n e^n}{n^5} = \lim_{n \to \infty} \frac{e^n}{n^4} \\
= \lim_{n \to \infty} \frac{n}{4 \log (n)} \\
\text{ Using L'hospital } \\
= \lim_{n \to \infty} \frac{1}{\frac{1}{n}}\\
= \lim_{n \to \infty} n = \infty
\end{gather*}
As it can be seen this assertion does not hold either.
