
\section{Exercise 1}\
Let
\begin{gather*}
H = I - 2 uu^* \\
|| u || = 1 \\
Hx = - \alpha e_1 \\
Hx = x- 2( u^*x)u = -\alpha e_1
\end{gather*}
Since $H$ is an orthonormal matrix:
\begin{gather*}
||x|| = || Hx|| = |\alpha| \\
\alpha = \pm ||x||\\
x^Hx = ||x||2 -2(u^*x)^2 = - \alpha e_1^T x\\
\end{gather*}
So we can get:
\begin{gather*}
u^*x = \sqrt{||x||^2 \pm ||x|| e_1^* x}
\end{gather*}
And set $\alpha = \text{sgn}(e_1^*x) ||x||$.
Thus:
\begin{gather*}
u^*x = \sqrt{\frac{1}{2} ||x||(||x|| + ||x_1||}\\
u = \frac{x+\alpha e_1}{2(u^* x)}
\end{gather*}
This method is better for constructing householder matrices because we dont have any special case for complex numbers since the inner product is real.

\section{Exercise 9}
We simply want to minimize our given formula for this exercise. 
The task is to maximize $t$ for $f(t) = || u - tx||_2$.
\begin{gather*}
\frac{\partial}{\partial x_k} \|\mathbf{x}\|_2 = \frac{x_k}{\|\mathbf{x}\|_2}\\
||u-tx||^2 \frac{\partial}{\partial t} = -2x || u -tx || Norm^{'}(u-tx) = \\
- 2x || u -tx || \frac{u-tx}{|| u-tx ||} = -2x( u-tx) \\
\end{gather*}
Searching for a minimum:
\begin{gather*}
-2x( u-tx) = 0 \Rightarrow t = u x^{*}
\end{gather*}
Now we need to verify that this is a minimum.
\begin{gather*}
f^{''}(t) = \frac{\partial}{\partial t} -2x( u-tx) = 2x^2 \Rightarrow 2x^2 > 0 \Rightarrow \text{ Minimum }
\end{gather*}
As we can see since the second derivative is always $>0$, we found a minimum for $f(t)$ at $u x^{*}$.
\section{Exercise 16}
We need to find the QR of $\left( \begin{array}{cc}
0 & -4\\
0 & 0 \\
-5 & -2 \\
\end{array} \right)$.
We recall the algorithm:
\begin{gather*}
 \alpha = - \mathrm{e}^{\mathrm{i} \arg x_k} \|\mathbf{x}\| \\
 \mathbf{u} = \mathbf{x} - \alpha\mathbf{e}_1 \\
 e_1 = \left(1 , \ldots \right)\\
 \mathbf{v} = {\mathbf{u}\over\|\mathbf{u}\|}\\
 Q = I - 2 \mathbf{v}\mathbf{v}^T
\end{gather*}
%Fuck yourself hard in the ass
\begin{gather*}
a_1 = \left( \begin{array}{cc}
0 \\
0\\
-5
\end{array} \right), 
||a_1|| e_1 = \left( \begin{array}{cc}
5 \\
0\\
0
\end{array} \right),
\alpha = -5 \\
u = \left( \begin{array}{cc}
0 \\
0\\
-5
\end{array} \right) +
(-5) 
\left( \begin{array}{cc}
1 \\
0\\
0
\end{array} \right) = 
\left( \begin{array}{cc}
-5 \\
0\\
-5
\end{array} \right)\\
v = \frac{u}{||u||} = \frac{1}{\sqrt{2}}\left( \begin{array}{cc}
1 \\
0\\
1
\end{array} \right),
H = I - 2 v^*v = \left( \begin{array}{ccc}
1 & 0 & 0 \\
0& 1 & 0\\
0 & 0 & 1
\end{array} \right) - 
\frac{2}{\sqrt{2}\sqrt{2}}
\left( \begin{array}{ccc}
1 & 0 & 1 \\
0& 0 & 0\\
1&0&1
\end{array} \right)\\
H = Q_1 = \left( \begin{array}{ccc}
0 & 0 & -1 \\
0 & 1 & 0\\
-1 & 0 & 0
\end{array} \right) \\
Q_1 A = \left( \begin{array}{cc}
5&2\\
0&0\\
0&4
\end{array} \right),
\end{gather*}
Now we do the second iteration:
\begin{gather*}
a_2 =\left( \begin{array}{c}
0 \\ 
4
\end{array} \right),
\alpha = -4 , 
u = \left( \begin{array}{c}
-4 \\ 
4
\end{array} \right)\\
v = \frac{1}{\sqrt{2}}\left( \begin{array}{c}
-1 \\ 
1
\end{array} \right), Q_2 = \left( \begin{array}{ccc}
1 & 0 & 0 \\ 
0 & 0 & 1\\
0 & 1 & 0
\end{array} \right)
\end{gather*}
So we can compute Q and R :

\begin{gather*}
Q = Q_1^*Q_2^* = \left( \begin{array}{ccc}
0 & 1 & 0 \\ 
0 & 0 &1 \\
1 & 0 & 0
\end{array} \right),
R = Q^*A = \left( \begin{array}{ccc}
-5 & 2 \\
0 & -4\\
0 & 0
\end{array} \right)
\end{gather*}

\section{Exercise 17}
We need to proof that $|| x||_2 = ||Qx||_2 $ and $ \langle x, y \rangle = \langle Qx, Qy \rangle$.
We can proof the inner product:
\begin{gather*}
\langle Qx,Qy \rangle = (Qx)^*Qy \\
= x^* Q^* Q y \\
= x^* y \\
= \langle x,y \rangle
\end{gather*}
Which means that $Q$ preserves the length of the vectors.
\begin{gather*}
||Qx||_2^2 = (Qx)^* Qx \\
= x^*Q^* Q x\\
= x^*x\\
= || x||_2^2 \\
\Rightarrow ||Qx||_2 = ||x||_2 
\end{gather*}
\section{Exercise 19}
%Bullshit here
We can compute the derivative of the following function, to show that $F(x) = ||Ax - b||_2^2 + \alpha ||x||_2^2$ minimizes $(A^TA + \alpha I)x = A^Tb$.
\begin{gather*}
\lim_{h \to 0} \dfrac{F(x+h) - F(x)}{h} \\
\lim_{h \to 0} \dfrac{F(x) + (Ah)^T Ah + \alpha h^Th - F(x)}{h}\\
\lim_{h \to 0} \dfrac{(Ah)^T Ah + \alpha h^Th}{h}\\
\lim_{h \to 0} \dfrac{||Ah||_2^2 + \alpha ||h||_2^2}{h}
\end{gather*}
Since the limit goes to zero, we can see that the function above minimizes $(A^TA + \alpha I)x = A^Tb, \forall x$.
\section{Exercise 33}
From the matrices we can write out the equation system
\begin{gather*}
3x + 2y = 3 \\
2x + 3y = 0\\
x+ 2y = 0
\end{gather*}
We define a function $S(x,y)$ which is out least squares function.
\begin{gather*}
S(x,y) = \left( 3 - \left( 3x + 2y \right) \right)^2 + \left( 0- \left( 2x+3y \right) \right)^2 + \left( 1 - \left( x + 2y \right) \right)^2 = 
14 x^2 + 17 y^2 + 28xy -20x -16y + 10 \\
\frac{\partial S}{\partial x} = 28x + 28y -20 = 0 = S_1 \\
\frac{\partial S}{\partial y} = 34y + 28x -16 = 0 = S_2\\
S_2 - S_1 = 6y +4 =0\\
y = -\frac{2}{3},\\
x = \frac{29}{21}
\end{gather*}