%Page 71
\section{Exercise 4}
We have the following newton iteration formula given:
\begin{gather*}
x_{n+1} = x_n - \frac{f(x_n)}{g(x_n)}\\
g(x) = \frac{f(x+f(x)) - f(x)}{f(x)}
\end{gather*}
Let $r$ be a solution such that $f(r)=0$ and $f'(r) \neq 0$. Moreover assume $|f''|$ is bounded in a neighborhood of $r$. We shall show that Newtons method is quadratically convergent provided that $x_0$ is sufficiently close to $r$.
Consider the function:
\begin{gather*}
k(x,\epsilon,\eta) = \frac{f''(\epsilon)(f'(x)  -\frac{1}{2} f''(\eta)(x-r)) + f''(\eta) )}{2 f'(x) + f''(\epsilon)f(x)}
\end{gather*}
By assumption the limit supremum of $|k(x,\epsilon,\eta)|$ exists as $x \to r$, $\epsilon \to r$ and $\eta \to r$. Let $C$ denote this limit and choose $\delta_1 > 0$ so small that $|k(x,\epsilon,\eta)|\leq 2C$ for every $x,\epsilon,\eta$ such that $|x-r| < \delta_1, |\epsilon -r | < \delta_1 , |\eta-r| < \delta_1$. Choose $\delta_2 > 0$ so small that $|x-r| < \delta_2 $ implies $2|f(x)| \leq \delta_1$.
Let:
\begin{gather*}
\delta = \min \left\lbrace \frac{\delta}{2},\delta_2 , \frac{1}{2C}\right\rbrace
\end{gather*}
Claim that if $x_o \in (r-\delta,r+\delta)$ then $x_n$ defined according to Newtons method satisfied $x_n \in ( r- \delta ,r+\delta)$ for all $n \in N$.
By Taylors theorem there exists $\epsilon_n$ between $x_n$ and $x_n + f(x_n)$ such that:
\begin{gather*}
f(x_n + f(x_n)) = f(x_n) + f'(x_n)f(x_n) + \frac{1}{2} f''(\epsilon_n)f(x_n)^2\\
\therefore g(x_n) = f'(x_n) + \frac{1}{2} f''(\epsilon_n)f(x_n)
\end{gather*}
It follows that:
\begin{gather*}
e_{n+1} = x_{n+1} -r = e_{n} - \frac{f(x_n)}{f'(x_n) + \frac{1}{2} f''(\epsilon_n)f(x_n)}
\end{gather*}
where $e_n = x_n -r$.
Using Taylor again, yields:
\begin{gather*}
0 = f(r) = f(x_n -e_n) = f(x_n) = f'(x_n)e_n + \frac{1}{2} f''(\eta_n)e_n^2
\end{gather*}
Fro some choice of $\eta_n$ between $r$ and $x_n$, Therefore:
\begin{gather*}
e_{n+1} = e_n - e_n\frac{f'(x_n) + \frac{1}{2} f''(\eta_n) e_n}{f'(x_n) + \frac{1}{2} f''(\epsilon_n)f(x_n)}\\
= e_n - e_n\frac{f'(x_n) -\frac{1}{2}f''(\epsilon_n)f(x_n) + \frac{1}{2} f''(\epsilon_n)f(x_n) + \frac{1}{2} f''(\eta_n) e_n}{f'(x_n) + \frac{1}{2} f''(\epsilon_n)f(x_n)}\\
= -e_n \frac{f''(\epsilon_n) f(x_n) + f''(\eta_n) e_n}{2f'(x_n) + f''(\epsilon_n)f(x_n)}\\
= -e_n \frac{f''(\epsilon_n)(f'(x_n)e_n - \frac{1}{2} f''(\eta_n) e_n^2)+ f''(\eta_n)e_n}{2f'(x_n) + f''(\epsilon_n) f(x_n)}\\
=-e_n^2 \frac{f''(\epsilon_n)(f'(x_n) - \frac{1}{2} f''(\eta_n)e_n) + f''(\eta_n)}{2f'(x_n) + f''(\epsilon_n)f(x_n)}\\
= -\epsilon_n^2 k(x_n,\epsilon_n,\eta_n)
\end{gather*}
Claim that $x_n\in (r-\delta,r+\delta)$. Since $\epsilon_n$ is between $x_n$ and $x_n+ f(x_n)$ then:
\begin{gather*}
|\epsilon_n -r| \leq | \epsilon_n - x_n | + |x_n - r| \leq |f(x_n) | + |x_n-r| < \delta_2 + \delta \leq \delta_1
\end{gather*}
Since $\eta_n$ is between $x$ and $x_n$:
\begin{gather*}
|\eta_n -r| \leq \delta < \delta_1
\end{gather*}
Thus $x_n, \epsilon_n, \eta_n$ satisfy $|x_n-r| < \delta_1, | \epsilon_n -r| < \delta_1 , | \eta_n -r | < \delta_1$, so consequently $|k(x_n,\epsilon_n,\eta_n)| < 2C$, therefore:
\begin{gather*}
|x_{n+1} -r| = |e_{n+1}|	 \leq 2C e_n^2 = 2C |x_n -r|^2 \leq 2C\delta^2 < \delta
\end{gather*}
This shows that $x_{n+1} \in ( r-\delta,r+\delta)$ and completes the induction. Moreover we have shown that $x_n$ is quadratically convergent.
\section{Exercise 7}
We need to find out the corresponding function for the newton iteration, for:
\begin{gather*}
x_{n+1} = 2x_n - x_n^2 y\\
x_{n+1} = x_n - \frac{f(x_n)}{f'(x_n)}\\
2x_n - x_n^2 y = x_n - \frac{f(x_n)}{f'(x_n)}\\
f(x_n) = (-x_n + x_n^2y) f'(x_n)
\end{gather*}

\section{Exercise 37}
\paragraph{a}
We use the non-linear gauss method to compute two iterations of the function:
\begin{gather*}
\begin{cases}
f_1 (x_1,x_2) = 4x_1^2 - x_2^2 \\
f_2 (x_1,x_2) = 4x_1x_2^2 -x_1 -1
\end{cases}
\end{gather*}
We compute the general Jacobi matrix:
\begin{gather*}
J = \left( \begin{array}{cc}
8x_1 & 4x_2 \\
4x_2^2 & 8x_1x_2
\end{array} \right),
J^{-1} = \frac{1}{16x_1^2 - 2x_2^2}\left( \begin{array}{cc}
2x_1 & -\frac{1}{2}\\
-x_2 & \frac{2x_1}{x_2}
\end{array} \right)
\end{gather*}
So we can calculate our first iteration, using $x_1 = 0, x_2 = 1$:
\begin{gather*}
\left( \begin{array}{c}
h_1\\
h_2
\end{array} \right) = 
\left( \begin{array}{cc}
0 & -\frac{1}{4}\\
-\frac{1}{2} & 0
\end{array} \right)
\left( \begin{array}{c}
-1 \\ 
-1
\end{array} \right) \Rightarrow h_1 = \frac{1}{4}, h_2 = \frac{1}{2}\\
\left( \begin{array}{c}
x_1^{(k+1)}\\
x_2^{(k+1)}
\end{array} \right) = 
\left( \begin{array}{c}
0\\
1
\end{array} \right)+
\left( \begin{array}{c}
\frac{1}{4} \\
\frac{3}{2}
\end{array} \right) \Rightarrow x_1^{(1)} = \frac{1}{4}, x_2^{(1)} = \frac{3}{2}
\end{gather*}
Now for the second iteration we still use the matrix $J^{-1}$, but we set $x_1 = \frac{1}{4}, x_2 = \frac{3}{2}$, so we get:
\begin{gather*}
\left( \begin{array}{c}
h_1\\
h_2
\end{array} \right) = 
\left( \begin{array}{cc}
\frac{2}{14} & -\frac{2}{14}\\
\frac{3}{14} & \frac{2}{21}
\end{array} \right)
\left( \begin{array}{c}
-2 \\ 
\frac{13}{4}
\end{array} \right) \Rightarrow h_1 = -\frac{3}{4}, h_2 = -\frac{5}{42}\\
\left( \begin{array}{c}
x_1^{(k+1)}\\
x_2^{(k+1)}
\end{array} \right) = 
\left( \begin{array}{c}
\frac{1}{4}\\
\frac{3}{2}
\end{array} \right) -
\left( \begin{array}{c}
\frac{3}{4} \\
\frac{5}{42}
\end{array} \right) \Rightarrow x_1^{(1)} = -\frac{1}{2}, x_2^{(1)} = \frac{57}{42}
\end{gather*}

\paragraph{b}

We use the non-linear gauss method to compute two iterations of the function:
\begin{gather*}
\begin{cases}
f_1 (x,y) = xy^2 + x^2 + x^4 -3 \\
f_2 (x,y) = x^3 y^5-2x^5y -x^2 +2 
\end{cases}
\end{gather*}
We compute the general Jacobi matrix:
\begin{gather*}
J = \left( \begin{array}{cc}
y^2 + 2xy + 4x^3 & 2xy + x^2 \\
2x^2y^5-10x^4y-2x & 5x^3 y^4 - 2x^5
\end{array} \right),\\
J^{-1} = \frac{1}{-8x^6 +x^4(20y^4 + 6y)+18x^3y^2 + 8x^2y^5 +x(y^6+2) + 4y}
\left( \begin{array}{cc}
5xy^4-2x^3 & \frac{-x-2y}{x}\\
-\frac{2(xy^5 - 5x^3y-1)}{x}& -\frac{-4x^3-2yx-y^2}{x^2}
\end{array} \right)
\end{gather*}
When we plug in the numbers, we get for $J^{-1}$:

\begin{gather*}
\frac{1}{51}\left( \begin{array}{cc}
3 & -3 \\
10 & 7
\end{array} \right)
\end{gather*}
Calculating the next iteration step:
\begin{gather*}
\left( \begin{array}{c}
h_1\\
h_2
\end{array} \right) = 
\left( \begin{array}{cc}
-\frac{3}{51} & \frac{3}{51}\\
-\frac{10}{51} & -\frac{7}{51}
\end{array} \right)
\left( \begin{array}{c}
0 \\ 
0
\end{array} \right) \Rightarrow h_1 = 0, h_2 = 0\\
\end{gather*}
Our calculations stop here, since we can see that the initial guess of $x=1,y=1$ is indeed the intersection between the two functions.

