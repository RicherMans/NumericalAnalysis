\section{Exercise 1}
We need to find the Schur decomposition of the given matrix $A$.
\paragraph{a}
\begin{gather*}
A = \left( \begin{array}{cc}
3 & 8 \\
-2 & 3 
\end{array} \right) \\
P^{*}AP = T
\end{gather*}
We first compute the eigenvalue and vectors of the given matrix $A$.
\begin{gather*}
\det( \lambda I - A ) = 0 = (\lambda -3 ) ^2 + 16 = 0 \\
= \lambda^2 - 6\lambda + 25 = 0 \\
\dfrac{6 \rpm \sqrt{-64}}{2} \rightarrow \lambda_1 = 3 + 4i \ \lambda_2 = 3 - 4i
\end{gather*}
The following eigenvectors result from the eigenvalues:
\begin{gather*}
v_1 = \left( \begin{array}{c}
-2i \\
1
\end{array} \right)
v_2 = \left( \begin{array}{c}
2 i \\
1
\end{array} \right)
\end{gather*}
We set $v_1 = w_1 = X_1$ and compute $w_2$ by using the given formula:
\begin{gather*}
w_{2}=X_{2}-\frac{w_{1}\cdot X_{2}}{\|w_{1}\|^{2}}w_{1} \\
w_2 =
\left( \begin{array}{c}
2i \\
1
\end{array} \right)
-
\dfrac{
\left( \begin{array}{c}
-2i \\
1
\end{array} \right)
\left( \begin{array}{c}
2i\\
1
\end{array} \right)
}{5} 
\left( \begin{array}{c}
-2i \\
1
\end{array} \right)
= 
\left( \begin{array}{c}
2i \\
1
\end{array} \right)
+ \frac{3}{5}
\left( \begin{array}{c}
-2i \\
1
\end{array} \right)\\
= 
\left( \begin{array}{c}
\frac{4}{5}i \\
\frac{8}{5}
\end{array} \right)\\
\end{gather*}
Further we compute the normed set of $\left[ \frac{w_1}{|w_1|}, \frac{w_2}{|w_2|} \right]$.
\begin{gather*}
\dfrac{w_1}{||w_1||} = \left( \begin{array}{c}
\frac{-2i}{\sqrt{5}} \\
\frac{1}{\sqrt{5}}
\end{array} \right)
, \dfrac{w_2}{||w_2||} = \left( \begin{array}{c}
\frac{i}{\sqrt{5}}\\
\frac{2}{\sqrt{5}}
\end{array} \right)
\end{gather*}
The set represents our transformation vector $P$, which is :
\begin{gather*}
P = \frac{1}{\sqrt{5}}\left( \begin{array}{cc}
-2i & i\\
1 & 2
\end{array} \right) \\
T = P^{*} A P =  \left( \begin{array}{cc}
3+4i & 6i\\
0 & 3-4i \\
\end{array} \right)
\end{gather*}
So we can decompose $A$ into $PTP^{*}$.

\paragraph{b}
\begin{gather*}
A = \left( \begin{array}{cc}
4 & 7 \\
1 & 12 
\end{array} \right) \\
P^{*}AP = T
\end{gather*}
We first compute the eigenvalue and vectors of the given matrix $A$.
\begin{gather*}
\det( \lambda I - A ) = 0 = (\lambda - 4 ) (\lambda -12) -7 = 0 \\
= \lambda^2 - 16\lambda + 41 = 0 \\
8 \rpm \sqrt{23} \rightarrow \lambda_1 = 8 + \sqrt{23} \ \lambda_2 = 8 - \sqrt{23}
\end{gather*}
The following eigenvectors result from the eigenvalues:
\begin{gather*}
v_1 = \left( \begin{array}{c}
-4+\sqrt{23}  \\
1
\end{array} \right)
v_2 = \left( \begin{array}{c}
-4-\sqrt{23} \\
1
\end{array} \right)
\end{gather*}
We set $v_1 = w_1 = X_1$ and compute $w_2$ by using the given formula:
\begin{gather*}
w_{2}=X_{2}-\frac{w_{1}\cdot X_{2}}{\|w_{1}\|^{2}}w_{1} \\
w_2 =
\left( \begin{array}{c}
-4-\sqrt{23} \\
1
\end{array} \right)
-
\dfrac{
\left( \begin{array}{c}
-4+\sqrt{23} \\
1
\end{array} \right)
\left( \begin{array}{c}
-4-\sqrt{23}\\
1
\end{array} \right)
}{1+ \left(\sqrt{23} -4 \right)^2 } 
\left( \begin{array}{c}
-4+\sqrt{23} \\
1
\end{array} \right)
= \\
\left( \begin{array}{c}
-4-\sqrt{23} \\
1
\end{array} \right)
+ \frac{6}{1+ \left(\sqrt{23} -4 \right)^2}
\left( \begin{array}{c}
-4+\sqrt{23} \\
1
\end{array} \right)\\
= 
\text{ Something large }
\end{gather*}
Since Gram Schmitt gives us too large vectors here, we better use the simple:
\begin{gather*}
w_1 = \left( \begin{array}{c}
-4+\sqrt{23}  \\
1
\end{array} \right)
w_2 = \left( \begin{array}{c}
1  \\
4 - \sqrt{23}
\end{array} \right)
\end{gather*}


Further we compute the normed set of $\left[ \frac{w_1}{|w_1|}, \frac{w_2}{|w_2|} \right]$.
\begin{gather*}
\dfrac{w_1}{||w_1||} = \left( \begin{array}{c}
\frac{\sqrt(23) - 4}{\sqrt{1 + \left( \sqrt{23} -4 \right)^2}} \\
\frac{1}{\sqrt{1 + \left( \sqrt{23} -4 \right)^2}}
\end{array} \right)
, \dfrac{w_2}{||w_2||} = \left( \begin{array}{c}
\frac{1}{\sqrt{1 + \left( \sqrt{23} -4 \right)^2}}\\
\frac{4-\sqrt{23}}{\sqrt{1 + \left( \sqrt{23} -4 \right)^2}}
\end{array} \right)
\end{gather*}
The set represents our transformation vector $P$, which is :
\begin{gather*}
P = \frac{1}{\sqrt{1 + \left( \sqrt{23} -4 \right)^2}}\left( \begin{array}{cc}
\sqrt{23} -4 & 1\\
1 & 4-\sqrt{23}
\end{array} \right) \\
T = P^{*} A P =  \left( \begin{array}{cc}
\frac{2 \left( 1247+243 \sqrt{23} \right)}{2401}  & -\frac{12 \left( 107+17 \sqrt{23} \right)}{2041} \\
0 & \frac{2 \left( 465+29\sqrt{23} \right) }{2401} \\
\end{array} \right)
\end{gather*}
So we can decompose $A$ into $PTP^{*}$.


\section{Exercise 2}
By applying Gershgorins Theorem to $D + E$ , we see that the spectrum of $D+E$ is the union of the Gershgorin disks of $D+E$, that is:
\begin{gather*}
z \in C : | z - a_{ii} | \leq \sum\limits_{j=1 ,j \neq i}^n |a_{ij}|
\end{gather*}
We get:
\begin{gather*}
D \cap E = \lambda - \lambda_i - a_{ii} \leq \sum\limits_{j=1 ,j \neq i}^n |a_{ij} + a_{ij}| = 
2 \sum\limits_{j=1 ,j \neq i}^n | a_{ij}| 
\end{gather*}
Which means that within the intersection, all elements of $a$ are contained by $D \cap E$.
\section{Exercise 4}
We need to prove if $A$ is Hermitian then the deflation procedure will produce a Hermitian matrix.
The procedure can be broken down into:
\begin{gather*}
V = B U \\
B = \left( \begin{array}{cc}
1 & 0\\
0 & Q
\end{array} \right)\\
U = \text{ unitary } \\
Q = \text{ unitary } \rightarrow B^* = B \\
VAV^* = BUA(BU)^* = B (U A U^*) B^* \\
U A U^* = (U A U^*)^* = UAU^* = \text{ diagonal } \\
B (U A U^*) B^*  = B D B^* = (BDB^*)^* = BD^*B^* = V A V^*\\
\end{gather*}
As we can see $VAV^*$ is hermitian, since its conjugate transpose it itself again.
\section{Exercise 12}
We need to prove that if $(I-vv*) x = y$ then $<x,y>$ is real.
We can show:
\begin{gather*}
(I-vv*) = Q \\
Qx = y \\
x^{*} Q x = x^{*} y = <x,y>
\end{gather*}
As we can see, the problem is only to show that $Q$ is real, so that $x^* Q x $ is real.
We see that if $vv^*$ is real, it is obvious that $x,y$ and $Q$ are real.
Assume that $vv^*$ is complex, so the matrix which result will be real in the diagonal and complex on the off diagonal elements.
Since $vv^*$ is symmetric, the items $vv^{*}_{ij} = vv^{*}_{ji}$.
We can moreover observe that Schurs decomposition only works if $ I - vv^*$ is unitary, meaning that $ || vv^* || = 0 $ or $|| vv^* ||^2 =2 $.
The case of $|| vv^* ||^2 = 0$ is trivial, since the identity matrix is a non complex one, we can verify that $x^*Qx$ is real.
Otherwise, if $ || vv^* ||^2 = 2$, we can decompose $Q$ into a diagonal form: 
\begin{gather*}
B = PQP^{-1} \\
Q \sim B
\end{gather*}
Since $Q$ and $B$ are similar in that case, finding a suitable orthogonalization leads then to the following form:
\begin{gather*}
B = \left( \begin{array}{cccc}
a_{11}& & &\\
& a_{22}& &\\
& & \ddots &\\
& & & a_{nn}\\
\end{array} \right)\\
x*PQP^{-1}x = x*Bx = \\ \left(
x_1, x_2, x_3, \hdots, x_n \right)
\left( \begin{array}{cccc}
a_{11}& & &\\
& a_{22}& &\\
& & \ddots &\\
& & & a_{nn}\\
\end{array} \right)
\left( \begin{array}{c}
\bar{x_1} \\
\bar{x_2} \\
\bar{x_3} \\ 
\vdots \\
\bar{x_n}
\end{array} \right) =
\\
a_{11} x_1 \bar{x_1} + a_{22} x_2 \bar{x_2} + \ldots + a_{nn} x_n \bar{x_n}
\end{gather*}
As we can see the summations which will be proceeded cancel the complex terms out.
\begin{equation}
<x,y> = x^*Qx = \text{ real }
\end{equation}

\section{Exercise 14}
First we show that $||QA||_2 = ||A||_2$
\begin{gather*}
|| QA ||_2 = (QA)^*(QA) = \\
A^*Q^*QA = \\
A^*A=
||A||_2
\end{gather*}
Now we show that $||AQ||_2 = ||A||_2$
\begin{gather*}
||AQ||_2 = (AQ)^*(AQ) = \\
Q^*A^*AQ = \\
Q^* ||A||_2 Q = \\
\text{ Since the norm is a number } \\
||A||_2 Q^*Q = ||A||_2 I = ||A||_2 
\end{gather*}

\section{Exercise 29}
\begin{equation*}
A = \left( \begin{array}{ccc}
6 &2 &1\\
1&-5 &0 \\
2&1 &4
\end{array} \right)
\end{equation*}
The upper limit of the eigenvalues are bounded by Gershgorin's Theorem.
\begin{gather*}
\lambda \in C : | \lambda | \leq || A ||_{\infty}
\end{gather*}
We can calculate $||A||_{\infty}$:
\begin{gather*}
||A||_{\infty} = \max (6+2+1,1+|-5|+0, 2+1+4)= 9 \\ \rightarrow |\lambda| \leq 9
\end{gather*}
The lower bound can be found by using the following theorem:
\begin{gather*}
\lambda-a_{ii} \leq \sum\limits_i \neq j | a_{ij}|
\end{gather*}
We get the following equations:
\begin{gather*}
\lambda - a_{11} \leq 2+1 \\
\lambda - a_{22} \leq 1 \\
\lambda - a_{33} \leq 2+1 \\
\therefore \lambda \geq 1
\end{gather*}
Which concludes that $1 \leq \lambda \leq 9 $.